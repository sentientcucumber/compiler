% Created 2014-08-25 Mon 00:13
\documentclass[11pt]{article}
\usepackage[utf8]{inputenc}
\usepackage[T1]{fontenc}
\usepackage{fixltx2e}
\usepackage{graphicx}
\usepackage{longtable}
\usepackage{float}
\usepackage{wrapfig}
\usepackage{rotating}
\usepackage[normalem]{ulem}
\usepackage{amsmath}
\usepackage{textcomp}
\usepackage{marvosym}
\usepackage{wasysym}
\usepackage{amssymb}
\usepackage{hyperref}
\tolerance=1000
\author{Michael Hunsinger}
\date{\today}
\title{Assignment 1, Part II}
\hypersetup{
  pdfkeywords={},
  pdfsubject={},
  pdfcreator={Emacs 24.3.1 (Org mode 8.2.7c)}}
\begin{document}

\maketitle

\section{Extended Micro Grammer}
\label{sec-1}
Write an extended Micro Grammer to include the equality and exponentiation
operators. The following grammers are meant to supplement the existing 
grammer explained in class.\newline
We will need the following definitions that were already defined in class.
\begin{verbatim}
<expression> -> <primary> | <primary> <add op> <expression>
<primary>    -> LParen <expression> RParen
<primary>    -> Id
<primary>    -> IntLiteral
<add op>     -> PlusOp | MinusOp
\end{verbatim}

\subsection{Exponential}
\label{sec-1-1}
The exponentiation should have higher precedence than \texttt{PlusOp} and 
\texttt{MinusOp} and should group from right to left, that is, 
\texttt{A**B**C = A**(B**C)}.
\begin{verbatim}
<expression> -> <primary> | <primary> <exponent> LParen <expression> RParen
<exponent>   -> ExpOp
\end{verbatim}

\subsection{Equality}
\label{sec-1-2}
The equality operator should have lower precednce than plus and minus,
and should not group, that is, \texttt{(A = B = C)} is not allowed.
\begin{verbatim}
<bool expression> -> <expression> <bool> <expression>
<bool>            -> EqualityOp
\end{verbatim}
% Emacs 24.3.1 (Org mode 8.2.7c)
\end{document}
