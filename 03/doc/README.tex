% Created 2014-09-13 Sat 00:00
\documentclass[a4paper]{article}
\usepackage[utf8]{inputenc}
\usepackage[T1]{fontenc}
\usepackage{fixltx2e}
\usepackage{graphicx}
\usepackage{longtable}
\usepackage{float}
\usepackage{wrapfig}
\usepackage{rotating}
\usepackage[normalem]{ulem}
\usepackage{amsmath}
\usepackage{textcomp}
\usepackage{marvosym}
\usepackage{wasysym}
\usepackage{amssymb}
\usepackage{hyperref}
\tolerance=1000
\usepackage{titling}
\addtolength{\topmargin}{-0.75in}
\addtolength{\textheight}{1.25in}
\addtolength{\oddsidemargin}{-.75in}
\addtolength{\evensidemargin}{-.75in}
\addtolength{\textwidth}{1.75in}
\usepackage{paralist}
\let\itemize\compactitem
\let\description\compactdesc
\let\enumerate\compactenum
\author{Michael Hunsinger}
\date{\today}
\title{Assignment 2}
\hypersetup{
  pdfkeywords={},
  pdfsubject={},
  pdfcreator={Emacs 24.3.1 (Org mode 8.2.7c)}}
\begin{document}

\maketitle

\section{Change log}
\label{sec-1}
\begin{itemize}
\item Changed the Reader from the bufio package to the bytes package. The
bufio Reader is made to read streaming data and cannot perform seeks.
\item Changed \verb~compiler.go~ to a \verb~scanner.go~, so it is more appropriately
named for future implementations.
\item Added a \verb~parser.go~ file to implement a compiler's parser functionality.
\item Added the ability to pass in the file on the command line.
\end{itemize}

\section{Go Documentation}
\label{sec-2}
This implentation of the scanner uses Google's new language Go. There are
instructions on how to setup Go, a description of the file structure, and
how to compile and run the program.

\subsection{Install Go}
\label{sec-2-1}
Download the appropriate installation from Google's Go website,
\url{http://golang.org/doc/install}, there is additional documentation located
on website as well.

\subsection{Go's Workspace}
\label{sec-2-2}
Extract files from the tarball into the desired location. Inside the root
folder you will find four directories
\begin{itemize}
\item \verb~bin~ compiled executables, along with sample micro program files
\item \verb~doc~ documentation
\item \verb~pkg~ package objects (the compiler package is located in here)
\item \verb~src~ source files
\begin{itemize}
\item \verb~compiler~ source files pertaining to the compiler package
\item \verb~main~ source files pertaining to the main package (the driver file)
\end{itemize}
\end{itemize}

We must also setup the \verb~GOPATH~ to ensure proper compilation of the files.
Follow the steps below to set \verb~GOPATH~ in a *unix environment. 
\begin{verbatim}
$ cd ../01
$ export GOPATH=$HOME/your/path/here/01
\end{verbatim}

\subsection{Compiling Source Files}
\label{sec-2-3}
There are two steps to compile and the executable; building the compiler
package and then build the executable.
\begin{verbatim}
$ cd ../01
$ go build compiler
$ go install main
\end{verbatim}
Now there is an executable in the \verb~bin~ folder.

\subsection{Running the Program}
\label{sec-2-4}
You can run the executable that was compiled. Ensure you are in the 
directory where the \verb~sample.micro~ file is located.
\begin{verbatim}
$ cd ../02/bin
$ ./main sample.micro
\end{verbatim}
This will run scan the \verb~sample.micro~ file. There is also a \verb~sample2.micro~
file in the \verb~bin~ folder that uses some of the tokens found in the 
extended Micro language. If you wish to scan this file, you will need to
change file name in \verb~../src/main/main.go~ on line 18.

\section{Sample Input and Output}
\label{sec-3}
There are still some existing issues when outputting larger files (sample2.micro) for instance
doesn't print quite right. 

\begin{verbatim}
sample.micro input
BEGIN A := B + (72 - C); END

sample.micro output
<system goal> --> BeginSym <program> EndSym EofSym
<system goal> --> BeginSym <statement list> EndSym EofSym
<system goal> --> BeginSym <statement> EndSym EofSym  ahead  AssignOp
<system goal> --> BeginSym <ident> := <expression> ; EndSym EofSym
<system goal> --> BeginSym Id := <expression> ; EndSym EofSym
<system goal> --> BeginSym Id := <primary> <add op> <expression> ; EndSym EofSym
<system goal> --> BeginSym Id := <ident> <add op> <expression> ; EndSym EofSym
<system goal> --> BeginSym Id := Id <add op> <expression> ; EndSym EofSym
<system goal> --> BeginSym Id := Id PlusOp <expression> ; EndSym EofSym
<system goal> --> BeginSym Id := Id PlusOp <primary> ; EndSym EofSym
<system goal> --> BeginSym Id := Id PlusOp ( <expression> ) ; EndSym EofSym
<system goal> --> BeginSym Id := Id PlusOp ( <primary> <add op> <expression> ) ; EndSym EofSym
<system goal> --> BeginSym Id := Id PlusOp ( IntLiteral <add op> <expression> ) ; EndSym EofSym
<system goal> --> BeginSym Id := Id PlusOp ( IntLiteral MinusOp <expression> ) ; EndSym EofSym
<system goal> --> BeginSym Id := Id PlusOp ( IntLiteral MinusOp <primary> ) ; EndSym EofSym
<system goal> --> BeginSym Id := Id PlusOp ( IntLiteral MinusOp <ident> ) ; EndSym EofSym
<system goal> --> BeginSym Id := Id PlusOp ( IntLiteral MinusOp Id ) ; EndSym EofSym
\end{verbatim}
% Emacs 24.3.1 (Org mode 8.2.7c)
\end{document}
